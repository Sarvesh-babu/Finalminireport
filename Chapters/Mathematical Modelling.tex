	\chapter{Mathematical Modelling}
	\label{chap:mathmodel}
	
	\section{SoC Calculation}
	
	
		
	The SoC of the vehicle is calculated from the following equations:
	

			\begin{equation}
		                             $SoC_{min}$ $\leq$ $SoC$ $\leq$ $SoC_{max}$\label{eq:socminmax}
		    \end{equation}

		    \begin{equation}                         
		               $SoC_{y}$ = $SoC_{y-1}$ + $P_{batt}(y)$ $\times$ $\partial y$ $\times$ $\eta c$
			\end{equation}
		    \begin{equation}             
		               $SoC_{y}$ = $SoC_{y-1}$ - $P_{batt}(y)$ $\times$ $\partial y$ $\times$ $\eta d$
			\end{equation}
		    \begin{equation}             
		               $P_{batt}(y)$ = $SoC_{y}$ $\times$ $E$
		    \end{equation}
		  Initial Power = Generation - Load


	  SoC limits:  
	 
	   $SoC_{min}$ and $SoC_{max}$ are  the maximum and minimum SoC of the EV respectively.
	   This constraint allows the SoC to vary between predefined minimum and maximum SoC.               
	
	\section{Best Pattern for charging}
	
	
	The charging pattern is determined by comparing the Energy required to the Real Time Price and by  identifying the minimum of it. 
	
	\begin{center}
		
		$T_{n}$ = $\sum^{24}_{i=1}$ ($Ch_{t} $ $\times$ [1 || 0 || -1 ] ) $\ast $ $Rtp_{i}$ 
    \end{center}	 
	\section{Maximum Power required by EV}
	
	
	Maximum Power demand occurs when all the three vehicles loads are high and the time block of maximum demand is identified. 
	
	\begin{center}
		
		$P_{t(total)}$ = $P_{t(car)}$ + $P_{t(truck)}$ + $P_{t(bus)}$
		
		$P_{t(total)}$ = argmax $\pi_{i}^{24}$ $\ast$ $P_{t(total)}$
	\end{center} 